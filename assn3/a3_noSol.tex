\documentclass{article}

\usepackage{etoolbox}


\usepackage{fullpage}
\usepackage{color}
\usepackage{amsmath}
\usepackage{url}
\usepackage{verbatim}
\usepackage{graphicx}
\usepackage{parskip}
\usepackage{amssymb}
\usepackage{listings} % For displaying code

\begin{document}

\definecolor{blu}{rgb}{0,0,1}
\def\blu#1{{\color{blu}#1}}
\definecolor{gre}{rgb}{0,.5,0}
\def\gre#1{{\color{gre}#1}}
\definecolor{red}{rgb}{1,0,0}
\def\red#1{{\color{red}#1}}
\def\norm#1{\|#1\|}
\newcommand{\argmin}[1]{\mathop{\hbox{argmin}}_{#1}}
\newcommand{\argmax}[1]{\mathop{\hbox{argmax}}_{#1}}
\def\R{\mathbb{R}}
\newcommand{\fig}[2]{\includegraphics[width=#1\textwidth]{a3f/#2}}
\newcommand{\centerfig}[2]{\begin{center}\includegraphics[width=#1\textwidth]{a3f/#2}\end{center}}
\def\items#1{\begin{itemize}#1\end{itemize}}
\def\enum#1{\begin{enumerate}#1\end{enumerate}}
\def\argmax{\mathop{\rm arg\,max}}
\def\argmin{\mathop{\rm arg\,min}}
\def\half{\frac 1 2}
\newcommand{\code}[1]{\lstinputlisting[language=Matlab]{a3f/#1}}
\newcommand{\alignStar}[1]{\begin{align*}#1\end{align*}}
\newcommand{\mat}[1]{\begin{bmatrix}#1\end{bmatrix}}



\title{CPSC 540 Assignment 3 (due February 27)}
\author{Density Estimation and Project Proposal}
\date{}
\maketitle


\section{Discrete and Gaussian Variables}

\subsection{MLE for General Discrete Distribution}

Consider a density estimation task, where we have two variables ($d=2$) that can each take one of $k$ discrete values. For example, we could have
\[
X = \mat{1 & 3\\4 & 2\\$k$ & 3\\1 & $k-1$}.
\]
The likelihood for example $x^i$ under a general discrete distribution would be
\[
p(x^i_1, x^i_2 | \Theta) = \theta_{x_1^i,x_2^i},
\]
where $\theta_{c_1,c_2}$ gives the probability of $x_1$ being in state $c_1$ and $x_2$ being in state $c_2$, for all the $k^2$ combinations of the two variables. In order for this to define a valid probability, we need all elements $\theta_{c_1,c_2}$ to be non-negative and they must sum to one, $\sum_{c_1=1}^k\sum_{c_2=1}^k \theta_{c_1,c_2} = 1$.
\enum{
\item Given $n$ training examples, \blu{derive the MLE for the $k^2$ elements of $\Theta$}.
\item Because of the sum-to-1 constraint, there are only $(k^2 - 1)$ degrees of freedom in the discrete distribution, and not $k^2$. \blu{Derive the MLE for this distribution assuming that
\[
\theta_{k,k} = 1 - \sum_{c_1=1}^k\sum_{c_2=1}^{k}\mathcal{I}[c_1\neq k,c_2 \neq k]\theta_{c_1,c_2},
\]
}so that the distribution only has $(k^2-1)$ parameters.
\item If we had separate parameter $\theta_{c_1}$ and $\theta_{c_2}$ for each variables, a reasonable choice of a prior would be a product of Dirichlet distributions,
\[
p(\theta_{c_1},\theta_{c_2}) \propto \theta_{c_1}^{\alpha_{c_1} - 1}\theta_{c_2}^{\alpha_{c_2} - 1}.
\]
For the general discrete distribution, a prior encoding the same assumptions  would be
\[
p(\theta_{c_1,c_2}) \propto \theta_{c_1,c_2}^{\alpha_{c_1} + \alpha_{c_2} - 2}.
\]
\blu{Derive the MAP estimate under this prior}.
}
Hint: it is convenient to write the likelihood for an example $i$ in the form
\[
p(x^i | \Theta) = \prod_{c \in [k]^2}\theta_c^{\mathcal{I}[x^i = c]},
\]
where $c$ is a vector containing $(c_1,c_2)$, $[x^i = c]$ evaluates to 1 if all elements are equal, and $[k]^2$ is all ordered pairs $(c_1,c_2)$. You can use the Lagrangian to enforce the sum-to-1 constraint on the log-likelihood, and you may find it convenient to define $N_c = \sum_{i=1}^n \mathcal{I}[x^i = c]$.


\subsection{Generative Classifiers with Gaussian Assumption}

Consider the 3-class classification dataset in this image:
\centerfig{.4}{sample}
In this dataset, we have 2 features and each colour represents one of the classes. Note that the classes are highly-structured: the colours each roughly follow a Gausian distribution plus some noisy samples.

Since we have an idea of what the features look like for each class, we might consider classifying  inputs $x$ using a \emph{generative classifier}. In particular, we are going to use Bayes rule to write
\[
p(y=c|x,\Theta) = \frac{p(x| y=c, \Theta) \cdot p(y=c|\Theta)}{p(x|\Theta)},
\]
where $\Theta$ represents the parameters of our model. To classify a new example $\hat{x}$, generative classifiers would use
\[
\hat{y} = \argmax_{y \in \{1,2,\dots,k\}} p(\hat{x}| y=c,\Theta)p(y=c|\Theta),
\]
where in our case the total number of classes $k$ is $3$ (The denominator $p(\hat{x}|\Theta)$ is irrelevant to the classification since it is the same for all $y$.)
% and $\theta_c$ is the set of parameters associated class $c$.
Modeling $p(y=c|\Theta)$ is eays: we can just use a $k$-state categorical distribution,
\[
p(y = c | \Theta) = \theta_c,
\]
where $\theta_c$ is a single parameter for class $c$. The maximum likelihood estimate of $\theta_c$ is given by $n_c/n$, the number of times we have $y^i = c$ (which we've called $n_c$) divided by the total number of data points $n$.

Modeling $p(x | y =c, \Theta)$ is the hard part: we need to know the \emph{probability of seeing the feature vector $x$ given that we are in class $c$}. This corresponds to solving a density estimation problem for each of the $k$ possible classes. 
To make the density estimation problem tractable, we'll assume that the distribution of $x$ given that $y=c$ is given by a $\mathcal{N}(\mu_c,\Sigma_c)$ Gaussian distribution for a class-specific $\mu_c$ and $\Sigma_c$,
\[
p(x | y=c, \Theta) = \frac{1}{(2\pi)^{\frac{d}{2}}|\Sigma_c|^{\half}}\exp\left(-\half (x-\mu_c)^T\Sigma_c^{-1}(x-\mu_c)\right).
\]
Since we are distinguishing between the probability under $k$ different Gaussians to make our classification, this is called \emph{Gaussian discriminant analysis} (GDA). In the special case where we have a constant $\Sigma_c = \Sigma$ across all classes it is known as \emph{linear discriminant analysis} (LDA) since it leads to a linear classifier between any two classes (while the region of space assigned to each class forms a convex polyhedron as in $k$-means clustering). Another common restriction on the $\Sigma_c$ is that they are diagonal matrices, since this only requires $O(d)$ parameters instead of $O(d^2)$ (corresponding to assuming that the features are independent univariate Gaussians given the class label).
Given a dataset $\mathcal{D}=\{(x^i, y^i)\}_{i=1}^n$, where $x^i\in\R^d$ and $y^i\in\{1,\ldots,k\}$, the maximum likelihood estimate (MLE) for the $\mu_c$ and $\Sigma_c$ in the GDA model is the solution to
\[
\argmax_{\mu_1,\mu_2,\dots,\mu_k,\Sigma_1,\Sigma_2,\dots,\Sigma_k} \prod_{i=1}^n p(x^i | y^i, \mu_{y^i},\Sigma_{y^i}).
\]
This means that the negative log-likelihood will be  equal to
\alignStar{
- \log p(X|y,\Theta) & = -\sum_{i=1}^n \log p(x^i | y^i | \mu_{y^i},\Sigma_{y^i})\\
& = \sum_{i=1}^n \frac{1}{2}(x^i - \mu_{y^i})^T\Sigma_{y^i}^{-1}(x^i - \mu_{y^i}) + \half\sum_{i=1}^n \log|\Sigma_{y^i}| + \text{const.}
}

\enum{
\item \blu{Derive the MLE for the GDA model under the assumption of \emph{common diagonal covariance} matrices}, $\Sigma_c = D$ ($d$ parameters). (Each class will have its own mean $\mu_c$.)
\item \blu{Derive the MLE for the GDA model under the assumption of \emph{individual scale-identity} matrices}, $\Sigma_c = \sigma_c^2 I$ ($k$ parameters).
\item It's painful to derive these from scratch, but you should be able to see a pattern that would allow other common restrictions. Without deriving the result from scratch (hopefully), \blu{give the MLE for the case of \emph{individual full} covariance matrices}, $\Sigma_c$ ($O(kd^2)$ parameters).
\item When you run \emph{example\_generative} it loads a variant of the dataset in the figure that has 12 features and 10 classes. This data has been split up into a training and test set, and the code fits a $k$-nearest neighbour classifier to the training set then reports the accuracy on the test data ($\sim 36\%$). The $k$-nearest neighbour model does poorly here since it doesn't take into account the Gaussian-like structure in feature space for each class label. Write a function \emph{generativeGaussian} that fits a GDA model to this dataset (using individual full covariance matrices). \blu{Hand in the function and report the test set accuracy}.
\item In this question we would like to replace the Gaussian distribution of the previous problem with the more robust multivariate-t distribution so that it isn't influenced as much by the noisy data.
Unlike the previous case, we don't have a closed-form solution for the parameters. However, if you run \emph{example\_tdist} it generates random noisy data and fits a multivariate-t model (you will need to add the \emph{minFunc} directory to the Matlab path for the demo to work). By using the \emph{multivariateT} model, write a new function \emph{generativeStudent} that implements a generative model that is based on the multivariate-t distribution instead of the Gaussian distribution. \blu{Report the test accuracy  with this model.}
}
Hints: you will be able to substantially simplify the notation in parts 1-3 if you use the notation $\sum_{i \in y_c}$ to mean the sum over all values $i$ where $y^i = c$. Similarly, you can use $n_c$ to denote the number of cases where $y_i = c$, so that we have $\sum_{i \in y_c}1 = n_c$. Note that the determinant of a diagonal matrix is the product of the diagonal entries, and the inverse of a diagonal matrix is a diagonal matrix with the reciprocals of the original matrix along the diagonal. For part three you can use the result from class regarding the MLE of a general multivariate Gaussian. You may find it helpful to use the included \emph{logdet.m} function to compute the log-determinant in more numerically-stable way.



\subsection{Self-Conjugacy for the Mean Parameter}

If $x$ is distributed according to a Gaussian with mean $\mu$,
\[
x \sim \mathcal{N}(\mu,\sigma^2),
\]
and we assume that $\mu$ itself is distributed according to a Gaussian
\[
\mu \sim \mathcal{N}(\alpha,\gamma^2),
\]
then the posterior $\mu | x$ also follows a Gaussian distribution.\footnote{We say that the Gaussian distribution is the `conjugate prior' for the Gaussian mean parameter (we'll formally discuss conjugate priors later in the course). Another reason the Gaussian distribution is important is that is the only (non-trivial) continuous distribution that has this `self-conjugacy' property.} \blu{Derive the form of the (Gaussian) distribution for $p(\mu|x,\alpha,\sigma^2,\gamma^2)$.} %You can assume that $\sigma=1$ and $\gamma=1$.

Hints: Use Bayes rule and use the $\propto$ sign to get rid of factors that don't depend on $\mu$. You can ``complete the square'' to make the product look like a Gaussian distribution, e.g. when you have $\exp(ax^2 - bx + \text{const})$ you can factor out an $a$ and add/subtract $(b/2a)^2$ to re-write it as
\begin{align*}
\exp\left(ax^2 - bx + const\right) & \propto
\exp\left(ax^2 - bx\right) = \exp\left(a(x^2 - (b/a)x)\right) \\& \propto \exp\left(a(x^2 - (b/a)x + (b/2a)^2)\right) =  \exp\left(a(x - (b/2a))^2\right).
\end{align*}
Note that multiplying by factors that do not depend on $\mu$ within the exponent does not change the distribution. In this question you will want to complete the square to get the distribution on $\mu$, rather than $x$.
You may find it easier to solve thie problem if you parameterize the Gaussians in terms of their `precision' parameters (e.g., $\lambda = 1/\sigma^2$, $\lambda_0 = 1/\gamma^2$) rather than their variances $\sigma^2$ and $\gamma^2$.



\section{Mixture Models and Expectation Maximization}

\subsection{Semi-Supervised Gaussian Discriminant Analysis}

Consider fitting a GDA model where some of the $y^i$ values are missing at random. In particular, let's assume we have $n$ labeled examples $(x^i,y^i)$ and then another another $t$ unlabeled examples $(x^i)$. This is a special case of \emph{semi-supervised learning}, and fitting generative models with EM is one of the oldest semi-supervised learning techqniues. When the classes exhibit clear structure in the feature space, it can be very effective even if the number of labeled examples is very small.

\enum{
\item \blu{Derive the EM update for fitting the parameters of a GDA model (with individual full covariance matrices) in the semi-superivsed setting where we have $n$ labeled examples and $t$ unlabeled examples}.
\item If you run the demo \emph{example\_SSL}, it will load a variant of the dataset from the previous question, but where the number of labeled examples is small and a large number of unlabeled examples are available. The demo first fits a KNN model and then a generative Gaussian model (once you are finished Question 1).
Because the number of labeled examples it quite small, the performance is worse than in Question 2. Write a function \emph{generativeGaussianSSL} that fits the generative Gaussian model of the previous question using EM to incorporate the unlabeled data. \blu{Hand in the function and report the test error when training on the full dataset}.
\item Repeat the previous part, but using the ``hard''-EM algorithm where we explicitly classify all the unlabeled examples. \blu{How does this change the performance and the number of iterations}?
}

Hint: for the first question most of the work has been done for you in the EM notes on the course webpage. You can use the result (**) and the update of $\theta_c$ from those notes, but you will need to work out the update of the parameters of the Gaussian distribution $p(x^i | y^i, \Theta)$.

Hint: for the second question, although EM often leads to simple updates, implementing them correctly can often be a pain. One way to help debug your code is to compute the observed-data log-likelihood after every iteration. If this number goes down, then you know your implementation has a problem. You can also test your updates of sets of variables in this way too. For example, if you hold the $\mu_c$ and $\Sigma_c$ fixed and only update the $\theta_c$, then the log-liklihood should not go down. In this way, you can test each of combinationso of updates on their to make sure they are correct.



\subsection{Mixture of Bernoullis}

The function \emph{example\_Bernoulli} loads a binarized version of the MNIST dataset and fits a density model that uses an independent Bernoulli to model each feature. If the reports the average NLL on the test data and shows 4 samples generated from the model. Unfortunately, the test NLL is infinity and the samples look terrible.
\enum{
\item To address the problem that the average NLL is infinity, modify the \emph{densityBernoulli} function to implement Laplace smoothing based on an extra argument $\alpha$. \blu{Hand in the code and report the average NLL with $\alpha = 1$}.
\item Write a new function implementing the mixture of Bernoullis model with Laplace smoothing of the $\theta$ values (note that Laplace smoothing only change the M-step). \blu{Hand in the code and report the average NLL with $\alpha = 1$ and $k=10$ for a particular run of the algorithm, as well as 4 samples from the model and 4 of the cluster images.}
}



\end{document}
%%% Local Variables:
%%% mode: latex
%%% TeX-master: "a3_aberk"
%%% End:
