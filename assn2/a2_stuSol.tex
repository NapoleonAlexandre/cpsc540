\documentclass{article}

\usepackage{etoolbox}


\usepackage{fullpage}
\usepackage{color}
\usepackage{amsmath}
\usepackage{url}
\usepackage{verbatim}
\usepackage{graphicx}
\usepackage{parskip}
\usepackage{amssymb}
\usepackage{listings} % For displaying code

\begin{document}

\definecolor{blu}{rgb}{0,0,1}
\def\blu#1{{\color{blu}#1}}
\definecolor{gre}{rgb}{0,.5,0}
\def\gre#1{{\color{gre}#1}}
\definecolor{red}{rgb}{1,0,0}
\def\red#1{{\color{red}#1}}
\def\norm#1{\|#1\|}
\newcommand{\argmin}[1]{\mathop{\hbox{argmin}}_{#1}}
\newcommand{\argmax}[1]{\mathop{\hbox{argmax}}_{#1}}
\def\R{\mathbb{R}}
\newcommand{\fig}[2]{\includegraphics[width=#1\textwidth]{a2f/#2}}
\newcommand{\centerfig}[2]{\begin{center}\includegraphics[width=#1\textwidth]{a2f/#2}\end{center}}
\def\items#1{\begin{itemize}#1\end{itemize}}
\def\enum#1{\begin{enumerate}#1\end{enumerate}}
\def\argmax{\mathop{\rm arg\,max}}
\def\argmin{\mathop{\rm arg\,min}}
\def\half{\frac 1 2}
\newcommand{\code}[1]{\lstinputlisting[language=Matlab]{a2f/#1}}
\newcommand{\alignStar}[1]{\begin{align*}#1\end{align*}}


\title{CPSC 540 Assignment 2 (due February 6)}
\author{Large-Scale Machine Learning}
\date{}
\maketitle

\setcounter{section}{-1}
\section{Unofficial Course Evaluation}

To help improve the course as we go along, or to suggest of how things could be done differently, please fill out the survey here:\\
\url{https://survey.ubc.ca/surveys/37-7974b781afc90962f53d98c2749/cpsc-540-informal-course-evaluation}

\section{Convergence Rates}

\subsection{Gradient Descent}

For minimizing a function $f$, in class we showed that if $\nabla f$ is Lipschitz continuous and $f$ is strongly-convex then gradient descent iterations,
\[
x^{t+1} = x^t - \alpha_t\nabla f(x^t),
\] 
with a step-size of $\alpha_t = 1/L$ satisfy
\[
f(x^t) - f(x^*) = O(\rho^t),
\]
for some $\rho < 1$ (we call this a linear convergence rate).
In this question you'll show some related properties.
\begin{enumerate}
\item The rate above is in terms of the function value $f(x^t)$, but we
  might also be interested in the convergence of the iterates $x^t$ to
  $x^*$. \blu{Show that if $f$ is differentiable and strongly-convex then a
    convergence rate of $O(\rho^t)$ in terms of the function values implies
    that the iterations have a convergence rate of
    \[
      \norm{x^t - x^*} = O(\rho^{t/2}).
    \]}

  \textsc{Solution}

  Since $f$ is differentiable and strongly convex we have, for any $x, y$
  \begin{align*}
    f(y) \geq f(x) + \nabla f(x)^T(y-x) + \frac{\mu}{2} \|y-x\|_2^2
  \end{align*}
  In particular, substituting for $x^*$ and $x^t$ it follows that
  \begin{align*}
    f(x^t) %
    \geq f(x^*) %
    + \underbrace{\nabla f(x^*)^T}_{=0} (x^t - x^*) %
    + \frac{\mu}{2} \|x^t - x^*\|_2^2 %
    = f(x^*) + \frac{\mu}{2} \|x^t - x^*\|_2^2
  \end{align*}
  Hence, using $f(x^t) - f(x^*) = O(\rho^t)$,
  \begin{align*}
    O(\rho^t) = f(x^t) - f(x^*) \geq \frac{\mu}{2} \|x^t - x^*\|_2^2  %
    \quad \implies \quad %
    \|x^t - x^*\|_2 \leq O(\rho^{t/2}).
  \end{align*}

  
\item Consider using a constant step-size $\alpha_t = \alpha$ for some positive
  constant $\alpha < 2/L$. \blu{Show that gradient descent converges linearly
    under this alternate step-size} (you can use the descent lemma).

  \textsc{Solution}

  Fix $\alpha < 2/L$ and define the iterates
  \begin{align*}
    x^{t+1} := x^t - \alpha \nabla f(x^t).
  \end{align*}
  The goal is to show that $f(x^t) - f(x^*) = O (\rho^t)$ for $\rho < 1$.  By
  the descent lemma,
  \begin{align*}
    f(y) - f(x) \leq \nabla f(x)^T (y-x) + \frac{L}{2} \|y - x\|_2^2. 
  \end{align*}
  Hence for $x := x^t$ and $y := x^{t+1}$ it follows that
  \begin{align*}
    f(x^{t+1}) - f(x^t) %
    &\leq \nabla f(x^t)^T (x^{t+1}-x^t) + \frac{L}{2}
      \|x^{t+1} - x^t\|_2^2 \\
    &= \big(\frac{\alpha^2 L}{2} - \alpha\big) \|\nabla f(x^t)\|_2^2
    \\
    &< - \alpha \|\nabla f(x^t)\|_2^2 + \alpha \|\nabla f(x^t)\|_2^2 %
      = 0 
  \end{align*}



\item In practice we typically don't $L$. A common strategy in this setting is
  to start with some small guess $L^0$ that we know is smaller than the true
  $L$ (usually we take $L=1$). On each iteration $t$, we initialize with
  $L^t = L^{t-1}$ and we check the inequality
  \[
    f\left(x^t - \frac{1}{L^t}\nabla f(x^t)\right) \leq f(x^t) -
    \frac{1}{L^t}\norm{\nabla f(x^t)}^2.
  \]
  If this is not satisfied, we double $L^t$ and test it again. This continues
  until we have an $L^t$ satisfying the inequality. \blu{Show that gradient
    descent with $\alpha_t = 1/L^t$ defined in this way has a linear
    convergence rate of
    \[
      f(x^t) - f(x^*) \leq \left(1 - \frac{\mu}{2L}\right)[f(x^0) - f(x^*).
    \]
  } Hint: if a function is $L$-Lipschitz continuous that it is also
  $L'$-Lipschitz continuous for any $L' \geq L$.
\item \blu{Describe a condition under which the step-sizes in the previous
    question would give a faster rate than $\rho = (1-\mu/L)$.}
\end{enumerate}

\subsection{Sign-Based Gradient Descent}

In some situations it might be hard to accurately compute the elements of the gradient, but we might have access to the sign of the gradient. For this setting, consider a sign-based gradient descent algorithm of the form
\[
x^{t+1} = x^t - \frac{\norm{\nabla f(x^t)}_1}{L}\text{sign}(\nabla f(x^t)),
\]
where we define the sign function element-wise as
\[
\text{sign}(x_j) = \begin{cases}+1 & x_j > 0\\0 & x_j =0\\-1 & x_j < 0\end{cases}.
\]
Consider an $f$ that is strongly-convex and is Lipschitz continuous in the $\infty$-norm, meaning that
\[
f(y) \leq f(x) + \nabla f(x)^T(y-x) + \frac{L_\infty}{2}\norm{y-x}_\infty^2,
\]
for all $y$ and $x$ and some $L_\infty$. 
\enum{
\item \blu{Show that the sign-based gradient descent method satisfies
\[
f(x^{t+1}) - f(x^*) \leq \left(1 - \frac{\mu}{L_\infty}\right)[f(x^t) - f(x^*)].
\]}
\item To compare this rate to the rate of gradient descent, we need to know the relationship between the usual Lipschitz constant $L$ (in the $2$-norm) and $L_\infty$. \blu{Show that the relationship between these constants is
\[
L_\infty \leq L \leq dL_\infty.
\]
}
}

\subsection{Block Coordinate Descent}

Consider a problem it makes sense to partition our variables into $k$ disjoint `blocks'
\[
x = \begin{bmatrix}x_1\\x_2\\x_3\\\vdots\\x_k\end{bmatrix},
\]
each of size $d/k$. Assume that $f$ is \emph{strongly-convex}, and \emph{blockwise strongly-smooth},
\[
\nabla^2 f(w) \succeq \mu I, \quad \nabla_{bb}^2 f(w) \preceq LI,
\]
for all $w$ and all blocks $b$.
Consider a \emph{block coordinate descent} algorithm where we use the iteration
\[
x^{t+1} = x^t - \frac{1}{L}(\nabla f(x^t)\circ e_{b_t}),
\]
where $b_t$ is the block we choose on iteration $t$, $e_b$ is vector of zeros with ones at the locations of block $b$, and $\circ$ means element-wise multiplication of the two vectors. (It's like coordinate descent except we're updating $d/k$ variables instead of just one.)

\enum{
\item Assume that we pick a random block on each iteration, $p(b_t = b) = 1/k$. \blu{Show that this method satisfies
\[
\mathbb{E}[f(x^{t+1})] - f(x^*) \leq \left(1-\frac{\mu}{Lk}\right)[f(x^t)-f(x^*)].
\]
}
\item Assume that each block $b$ has its own strong-smoothness constant $L_b$,
\[
\nabla_{bb}^2 f(w) \preceq L_bI,
\]
so that the strong-smoothness constant from part 1 is given by $L = \max_b\{L_b\}$. \blu{Show that if we sample the blocks proportional to $L_b$, $p(b_t = b) = \frac{L_b}{\sum_{b'}L_{b'}}$ and we use a larger step-size of $1/L_{b_t}$, then we obtain a faster convergence rate provided that some $L_b \neq L$.}
}


\section{Large-Scale Algorithms}

\subsection{Coordinate Optimization}

The function \emph{example\_logistic} loads a dataset and tries to fit an L2-regularized logistic regression model using coordinate optimization. Unfortunately, if we use $L_f$ as the Lipschitz constant of $\nabla f$, the runtime of this procedure is $O(d^3 + nd^2\frac{L_f}{\mu}\log(1/\epsilon))$. This comes from spending $O(d^3)$ computing $L_f$, having an iteration cost of $O(nd)$, and requiring a $O(d\frac{L_f}{\mu}\log(1/\epsilon))$ iterations. This non-ideal runtime is also reflected in practice: the algorithm's iterations are relatively slow and even after 500 ``passes'' through the data it isn't particularly close to the optimal function value.

\enum{
\item Modify this code so that the runtime of the algorithm is $O(nd\frac{L_c}{\mu}\log(1/\epsilon))$, where $L_c$ is the Lipschitz constant of \emph{all} partial derivatives $\nabla_i f$. You can do this by modifying the iterations so they have a cost $O(n)$ instead of $O(nd)$, and instead of using a step-size of $1/L_f$ they use a step-size of $1/L_c$ (which is given by $\frac{1}{4}\max_j\{\norm{x_j}^2\} + \lambda$).
 \blu{Hand in your code and report the final function value and total time}.
 \item To further improve the performance, make a new version of the code which samples the variable to update $j_t$ proportional to the individual Lipschitz constants $L_j$ of the coordinates, and use a step-size of $1/L_{j_t}$. You can use the function \emph{sampleDiscrete} to sample from discrete distribution given the probability mass function.  \blu{Hand in your code, and report the final function value as well as the number of passes}.
\item \blu{Report the number of passes the algorithm takes as well as the final function value} if you use \emph{uniform sampling} but use a step-size of $1/L_{j_t}$. 
\item Suppose that when we use a step-size of $1/L_{j_t}$, we see that uniform sampling outperforms Lipschitz sampling. \blu{Why would this be consistent with the bounds we stated in class?}
}


\subsection{Proximal-Gradient}


If you run the demo \emph{example\_group}, it will load a dataset and fit a multi-class logistic regression (softmax) classifier. This dataset is actually \emph{linearly-separable}, so there exists a set of weights $W$ that can perfectly classify the training data (though it may be difficult to find a $W$ that perfectly classifiers the validation data). However, 90\% of the columns of $X$ are irrelevant. Because of this issue, when you run the demo you find that the training error is $0$ while the test error is something like $0.2980$.

\enum{
\item Write a new function, \emph{softmaxClassifierL2}, that fits a multi-class logistic regression model with L2-regularization (this only involves modifying the objective function). \blu{Hand in the modified loss function and report the best validation error achievable with $\lambda = 10$ (which is best value among powers to 10). Also report the number of non-zero parameters in the model and the number of original features that the model uses}.
\item While L2-regularization reduces overfitting a bit, it still uses all the variables even though 90\% of them are irrelevant. In situations like this, L1-regularization may be more suitable. Write a new function, \emph{softmaxClassifierL1}, that fits a multi-class logistic regression model with L1-regularization. You can use the function \emph{proxGradL1}, which minimizes the sum of a differentiable function and an L1-regularization term.  \blu{Report the number of non-zero parameters in the model and the number of original features that the model uses}.
\item L1-regularization achieves sparsity in the \emph{model parameters}, but in this dataset it's actually the \emph{original features} that are irrelevant. We can encourage sparsity in the original features by using \emph{group} L1-regularization. Write a new function, \emph{proxGradGroupL1}, to allow (disjoint) \emph{group} L1-regularization. Use this within a new function, \emph{softmaxClassiferGL1}, to fit a group L1-regularized multi-class logistic regression model (where \emph{rows} of $W$ are grouped together and we use the L2-norm of the groups).  \blu{Hand in both modified  functions (\emph{softmaxClassifierGL1} and \emph{proxGradGroupL1}) and report the validation error achieved with $\lambda=10$. Also report the number of non-zero parameters in the model and the number of original features that the model uses}.
}



\subsection{Stochastic Gradient}

If you run the demo \emph{example\_stochastic}, it will load a dataset and try to fit an L2-regularized logistic regression model using 10 ``passes'' of stochastic gradient using the step-size of $\alpha_t = 1/\lambda t$ that is suggested in many theory papers. Note that the demo is quite slow as Matlab doesn't do well with `for' loops, but if you implemented this in C this would be very fast even though there are 50,000 training examples.

Unfortunately, even if we ignore the Matlab-slowness, the performance of this stochastic gradient method is atrocious. It often goes to areas of the parameter space with the objective function overflows and the final value is usually in the range of something like $6.5-7.5 \times 10^4$. This is quite far from the solution of $2.7068 \times 10^4$ and is even worse than just choosing $w=0$ which gives $3.5 \times 10^4$. (This is unlike gradient descent and coordinate optimization, which never increase the objective function.)

\enum{
\item Although $\alpha_t = 1/\lambda$ gives the best possible convergence rate in the worst case, in practice it's typically horrible (as we're not usually opitmizing the hardest possible $\lambda$-strongly convex function). Experiment with different choices of step-size to see if you can get better performance. \blu{Report the step-size that you found gave the best performance, and the objective function value obtained by this strategy for one run}.
\item Besides tuning the step-size, another strategy that often improves the performance is using a (possibly-weighted) average of the iterations $w^t$. Explore whether this strategy can improve performance.  \blu{Report the performance with an averaging strategy, and the objective function value obtained by this strategy for one run}. (Note that the best step-size sequence with averaging might be different than without averaging.)
\item A popular variation on stochastic is AdaGrad, which uses the iteration
\[
w^{t+1} = w^t - \alpha_t D_t \nabla f(x^t),
\]
where the element in position $(i,i)$ of the diagonal matrix $D_t$ is given by $1/\sqrt{\delta + \sum_{t=0}^t\nabla f(x^t)}$ (and we don't average the steps).  Implement this algorithm and experiment with the tuning parameters $\alpha_t$ and $\delta$. \blu{Hand in your code as well as the best step-size sequence you found and again report the performance for one run}.
\item Impelement the SAG algorithm with a step-size of $1/L$, where the $L$ is the maximum Lipschitz constant across the training examples ($L = 0.25\max_i\{\norm{x^i}^2\} + \lambda$).  \blu{Hand in your code and again report the performance for one run}.
}


\section{Kernels and Duality}

\subsection{Fenchel Duality}


Recall that the Fenchel dual for the primal problem
\[
P(w) = f(Xw) + g(w),
\]
is the dual problem
\[
D(z) = - f^*(-z) - g^*(X^Tz),
\]
or if we re-parameterize in terms of $-z$:
\begin{equation}
\label{eq:FD2}
D(z) = - f^*(z) - g^*(-X^Tz),
\end{equation}
where $f^*$ and $g^*$ are the convex conjugates.
Convex conjugates are discussed in Section 3.3 of Boyd and Vandenberghe (\url{http://stanford.edu/~boyd/cvxbook/bv_cvxbook.pdf}). Read this, then \blu{derive the Fenchel dual for the following problems}:
\begin{center}
\begin{tabular}{lll}
1. & $P(w) = \frac{1}{2}\norm{Xw - y}^2 +  \frac{\lambda}{2}\norm{w}^2$ & (L2-regularized least squares)\\
2. & $P(w) = \norm{Xw -y}_1 + \lambda\norm{w}_1$ & (robust regression with L1-regularization)\\
3. & $P(w) = \sum_{i=1}^N \log(1+\exp(-y^iw^Tx^i)) + \frac{\lambda}{2}\norm{w}^2$ & (regularized maximum entropy)
\end{tabular}
\end{center}
Hint: Don't try to take the Lagrangian dual, a generic strategy to compute the special case of Fenchel duals is as follows:
\begin{itemize}
\item Determine $X$, $f$, and $g$ to put the problem into the primal format.
\item Determine the form of $f^*$ and $g^*$ (note that $A$ here is not relevant).
\item Evaluate $f^*$ at $-z$ and $g^*$ at $X^Tz$ to get the final form.
\end{itemize}
For a differentiable $f$, you can often solve for the value achieving the $\sup$ inside of $f^*(v)$ by taking the gradient of $(x^Tv - f(x))$ and setting it to zero (keeping in mind whether there are values of $v$ where the $\sup$ might be infinity). Example 3.26 in the book gives the convex conjugate in the case where $f$ is a norm. Section 3.3.2 of the book shows how the convex conjugate changes if you scale a function and/or compose a function with an affine transformation. For parts 1 and 2, the $X$ in the primal will just be the data matrix $X$. But for part 3, it will be easier if you define $X$ as a matrix with row $i$ is given by $y^ix^i$. For part 3 you'll want to use $f(v) = \sum_{i=1}^n \log(1 + \exp(v_i))$, which is a \emph{separable} function (meaning that you can optimize each $z_i$ independently).

\textsc{Solution}



\subsection{Stochastic Dual Coordinate Ascent}
The dual of the SVM problem,
\[
P(w) = \sum_{i=1}^N \max\{0,1-y^iw^Tx^i\} + \frac{\lambda}{2}\norm{w}^2,
\]
is
\[
D(z) = e^Tz - \frac{1}{2\lambda}z^TYXX^TYz, \quad\text{s.t. } 0 \leq z_i \leq 1, \forall_i.
\]
where $e$ is a vector of ones, $Y$ is diagonal matrix with the $y^i$ values along the diagonal, and we have $w^* = \frac{1}{\lambda}X^TYz^*$. Starting from \emph{example\_dual.m}, implement a dual coordinate optimization  strategy to optimize the SVM objective. \blu{Hand in your code, report the optimal value of the primal and dual objectives with $\lambda = 1$, and report the number of support vectors}

Hint: the objective function is a quadratic and the constraints are just lower and upper bounds. This lets you  derive the optimal update for one variable with the other held fixed: solve for the value of $z_i$ with a partial derivative of zero, and if this violates the constraints then the solution must be either $z_i = 0$ or $z_i = 1$ (depending on which one is lower).

\subsection{Large-Scale Kernel Methods}

The function \emph{kernelRegression.m} implements kernel regression with the squared error, L2-regularizer, and Gaussian kernel. If you run your cross-validation code from Assignment 1 Question 1.2, you'll find that it achieves similar performance to using Gaussian RBFs.
\enum{
\item \blu{Report the $\lambda$ and $\sigma$ reported  using cross-validation on this previous assignment question. 
What are the (approximate) relationships between $\lambda$ and $\sigma$ between the two models} (the one with Gaussian RBFs and the other with Gaussian kernels).
\item Implement the \emph{subset of regressors} model for large-scale kernel methods we discussed in class. \blu{Hand in your code and report the qualitative performance (describe how well the model fits the data visually) for small and large values of the number of regressors $m$.}
\item Implement the \emph{random kitchen sink} model for large-scale kernel methods we discussed in class. \blu{Hand in your code and contrast the performance of this method with the subset of regressors model, for both large and small $m$.
}
}



\end{document}
%%% Local Variables:
%%% mode: latex
%%% TeX-master: t
%%% End:
